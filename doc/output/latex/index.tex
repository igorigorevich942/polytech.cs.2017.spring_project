Программа -\/ это реализация игры \char`\"{}Забег\char`\"{}.

\subsection*{Информация}

Проект написан на языке C с использование библиотеки allegro. Unit-\/тесты были написаны с истользованием фреймворка Unity.

\subsubsection*{Скриншот}



\subsubsection*{Правила игры}

Ваша задача -\/ как можно быстрее пересечь финишную черту.

\subsubsection*{Управление}

Для старта игры небходимо нажать клавишу \char`\"{}\+Enter\char`\"{}.

После старта игры для увеличения скорости игрока нужно нажимать клавиши \char`\"{}стрелка влево\char`\"{} и \char`\"{}стрелка вправо\char`\"{}, чем чаще нажимать, тем выше скорость.

После финиширования для выхода из игры небоходимо нажать кливишу \char`\"{}\+Enter\char`\"{}.

\subsubsection*{Структура директории}

Файлы в директории лежат следующим образом\+:

\tabulinesep=1mm
\begin{longtabu} spread 0pt [c]{*{2}{|X[-1]}|}
\hline
\rowcolor{\tableheadbgcolor}\textbf{ Каталог }&\textbf{ Описание  }\\\cline{1-2}
\endfirsthead
\hline
\endfoot
\hline
\rowcolor{\tableheadbgcolor}\textbf{ Каталог }&\textbf{ Описание  }\\\cline{1-2}
\endhead
src/ &файлы исходного кода \\\cline{1-2}
src/res &ресурсы игры \\\cline{1-2}
src/test &unit-\/тесты \\\cline{1-2}
doc/ &документация \\\cline{1-2}
doc/res/ &ресурсы для документации \\\cline{1-2}
\end{longtabu}
\subsubsection*{Сборка}

Для того, чтобы собрать проект напишите следующее\+: 
\begin{DoxyCode}
make
\end{DoxyCode}
 Чтобы восстановить все с нуля, выполните следующее действие\+: 
\begin{DoxyCode}
make clean
\end{DoxyCode}
 Для запуска тестов необходимо ввести следующую команду\+: 
\begin{DoxyCode}
make D\_UNITY=../Unity check
\end{DoxyCode}
 Для сборки документации\+: 
\begin{DoxyCode}
make doxygen
\end{DoxyCode}
 Для сборки документации в формате P\+DF\+: 
\begin{DoxyCode}
make pdf
\end{DoxyCode}
 Для сборки документации в формате H\+T\+ML\+: 
\begin{DoxyCode}
make html
\end{DoxyCode}


\subsection*{Авторы}


\begin{DoxyItemize}
\item {\bfseries Пупкин Васий} -\/ \href{mailto:pupkin@mail.ru}{\tt pupkin@mail.\+ru}
\end{DoxyItemize}

\subsection*{Лицензия}

This project is licensed under the M\+IT License -\/ see the \mbox{[}L\+I\+C\+E\+N\+SE\mbox{]}(L\+I\+C\+E\+N\+SE) file for details 